\documentclass[a4paper,10pt]{article}

% Packages
\usepackage[utf8]{inputenc}
\usepackage{hyperref}
\usepackage{caption}
\usepackage{color}
%\usepackage{parskip}
\usepackage[acronym, nonumberlist]{glossaries}	%Create clossaries and acronym tables
\usepackage{graphicx}
\usepackage{subcaption}
\usepackage{tikz}
\usepackage{booktabs}
\usepackage{datetime}
\usepackage{longtable}
\usepackage{tabularx}
\usepackage{tipa}
\usepackage{authblk}

\usepackage[binary-units=true]{siunitx}
\DeclareSIUnit{\foot}{ft}

% Listings
\usepackage{listings}
\lstset{ %
  backgroundcolor=\color{black},   % choose the background color; you must add \usepackage{color} or \usepackage{xcolor}; should come as last argument
  basicstyle=\footnotesize\color{white},        % the size of the fonts that are used for the code
  breakatwhitespace=false,         % sets if automatic breaks should only happen at whitespace
  breaklines=true,                 % sets automatic line breaking
  captionpos=b,                    % sets the caption-position to bottom
  commentstyle=\color{white},      % comment style
  deletekeywords={...},            % if you want to delete keywords from the given language
  escapeinside={\%*}{*)},          % if you want to add LaTeX within your code
  extendedchars=true,              % lets you use non-ASCII characters; for 8-bits encodings only, does not work with UTF-8
  frame=single,	                   % adds a frame around the code
  keepspaces=true,                 % keeps spaces in text, useful for keeping indentation of code (possibly needs columns=flexible)
  keywordstyle=\color{blue},       % keyword style
  language=Octave,                 % the language of the code
  morekeywords={*,...},            % if you want to add more keywords to the set
  numbers=left,                    % where to put the line-numbers; possible values are (none, left, right)
  numbersep=5pt,                   % how far the line-numbers are from the code
  numberstyle=\tiny\color{white},  % the style that is used for the line-numbers
  rulecolor=\color{black},         % if not set, the frame-color may be changed on line-breaks within not-black text (e.g. comments (green here))
  showspaces=false,                % show spaces everywhere adding particular underscores; it overrides 'showstringspaces'
  showstringspaces=false,          % underline spaces within strings only
  showtabs=false,                  % show tabs within strings adding particular underscores
  stepnumber=2,                    % the step between two line-numbers. If it's 1, each line will be numbered
  stringstyle=\color{white},       % string literal style
  tabsize=2,	                   % sets default tabsize to 2 spaces
  title=\lstname                   % show the filename of files included with \lstinputlisting; also try caption instead of title
} 


% Listings alternative; 
% needs to be invoked with -shell-escape; e.g.
%   pdflatex -shell-escape -synctex=1 -interaction=nonstopmode %source
% Requires: sudo apt install python-pygments
% Language: pygmentize -L lexers

%\usepackage{minted}

% Technologies
\newacronym{AQL}{AQL}{Array Querying Language}
\newacronym{STARE}{STARE}{Spatio-Temporal Adaptive-Resolution Encoding}
\newacronym{HTM}{HTM}{Hierarchical Triangular Mesh}
\newacronym{SQT}{SQT}{Sphere Quadtree}
\newacronym{QTM}{QTM}{Quaternary Triangular Mesh}
\newacronym{RLE}{RLE}{Run-Length Encoding}
\newacronym{HSTM}{HSTM}{Hierarchical Spherical Triangular Mesh}
\newacronym{DLF}{DLF}{Dense Load Format}
\newacronym{GCC}{GCC}{Google Cloud Computing}
\newacronym{VM}{VM}{Virtual Machine}
\newacronym{I2}{I2}{Integerized Latitude-Longitude}
\newacronym{AWS}{AWS}{Amazon Web Services}
\newacronym{NFS}{NFS}{Network File System}
\newacronym{OLFS}{OLFS}{OPeNDAP Lightweight Front-end Servlet}
\newacronym{BES}{BES}{Back End Server}
\newacronym{DOI}{DOI}{Digital Object Identifier}
\newacronym{THREDDS}{THREDDS}{Thematic Real-time Environmental Distributed Data Services}
\newacronym{TDS}{TDS}{\acrshort{THREDDS} Data Server}
\newacronym{DAP}{DAP}{Data Access Protocol}
\newacronym{RDBMS}{RDBMS}{Relational Database Management System}
\newacronym{FTP}{FTP}{File Transfer Protocol}
\newacronym{HPC}{HPC}{High Performance Computing}
\newacronym{LIDAR}{LIDAR}{Light Detection and Ranging}
\newacronym{AIS}{AIS}{Automatic identification system}
\newacronym{API}{API}{application programming interface}
\newacronym{DAS}{DAS}{Document attribute structure}
\newacronym{HDF}{HDF}{Hierarchical Data Format}
\newacronym{RO-RO}{RO-RO}{Roll-on/Roll-off}
\newacronym{GPS}{GPS}{Global Positioning System}
\newacronym{MMSI}{MMSI}{Maritime Mobile Service Identity}
\newacronym{SDSS}{SDSS}{Sloan Digital Sky Survey}
\newacronym{ETL}{ETL}{extract, transform, and load}
\newacronym{DGGS}{DGGS}{Discrete Global Grid Systems}
\newacronym{WCS}{WCS}{Web Coverage Service}
\newacronym{WMS}{WMS}{Web Map Service}
\newacronym{CSL}{CSL}{Citation Style Language}
\newacronym{OCCUR}{OCCUR}{OPeNDAP Citation Creator}
\newacronym{MODSCAG}{MODSCAG}{MODIS Snow-Covered Area and Grain size}
\newacronym{NDSI}{NDSI}{Normalized Difference Snow Index}
\newacronym{NDVI}{NDVI}{Normalized Difference Vegetation Index}
\newacronym{MESMA}{MESMA}{Multiple Endmember Spectral Mixture Models}
\newacronym{MEMSCAG}{MEMSCAG}{multiple endmember snow-covered area and grain size}
\newacronym{TM}{TM}{Thematic Mapper}
\newacronym{ETM}{ETM}{Enhanced Thematic Mapper}
\newacronym{L2G}{L2G}{Gridded Level-2}
\newacronym{SCAGD}{SCAGD}{Snow-Covered Area, God Damn}
\newacronym{HRPP}{HRPP}{High Resolution Precipitation Product}
\newacronym{STARS}{STARS}{Spatiotemporal Arrays, Raster and Vector Datacubes}
\newacronym{STDOS}{STDOS}{Simple Tile Database On Steroids}
\newacronym{AFL}{AFL}{Array Functional Language}
\newacronym{REST}{REST}{Representational state transfer}
\newacronym{RDF}{RDF}{Resource Description Framework}
\newacronym{GUI}{GUI}{Graphical User Interface}
\newacronym{GIS}{GIS}{Geograpic Information System}

% Projects/Missions
\newacronym{TRMM}{TRMM}{Tropical Rainfall Measuring Mission}
\newacronym{OSM}{OSM}{OpenStreetMap}
\newacronym{GOES}{GOES}{Geostationary Operational Environmental Satellite}
\newacronym{VIIRS}{VIIRS}{Visible Infrared Imaging Radiometer Suite}
\newacronym{AVIRIS}{AVIRIS}{Airborne Visible / Infrared Imaging Spectrometer}
\newacronym{AVIRIS-NG}{AVIRIS-NG}{\acrshort{AVIRIS} Next Generation}
\newacronym{MODIS}{MODIS}{Moderate Resolution Imaging Spectroradiometer}
\newacronym{QuickSCAT}{QuickSCAT}{Quick Scatterometer}
\newacronym{NPP}{NPP}{National Polar-orbiting Partnership}
\newacronym{DMSP}{DMSP}{Defense Meteorological Program}
\newacronym{OLS}{OLS}{Operational Line-Scan System}


% General
\newacronym{PI}{PI}{Principal Investigator}
\newacronym{CFR}{CFR}{Code of Federal Regulations}
\newacronym{SW}{SW}{Shortwave}
\newacronym{SWE}{SWE}{Snow Water Equivalent}
\newacronym{NIR}{NIR}{Near infrared}
\newacronym{SWIR}{SWIR}{Short Wave infrared}
\newacronym{FOV}{FOV}{Field of View}
\newacronym{FOSS}{FOSS}{Free and Open Source Software}



% Organizations
\newacronym{ESIP}{ESIP}{Earth Science Information Partners}
\newacronym{AGU}{AGU}{American Geophysical Union}
\newacronym{OPeNDAP}{OPeNDAP}{Open-source Project for a Network Data Access Protocol}
\newacronym{UCSB}{UCSB}{University of California, Santa Barbara}
\newacronym{SDSC}{SDSC}{San Diego Supercomputer Center}
\newacronym{ASO}{ASO}{Airborne Snow Observatory}
\newacronym{IMO}{IMO}{International Maritime Organization}
\newacronym{EU}{EU}{European Union}
\newacronym{SIRC}{SIRC}{Seafarers International Research Centre}
\newacronym{OGC}{OGC}{Open Geospatial Consortium}
\newacronym{RDA}{RDA}{Research Data Alliance}
\newacronym{SEDAC}{SEDAC}{NASA Socioeconomic Data and Applications Center}
\newacronym{BCO-DMO}{BCO-DMO}{The Biological and Chemical Oceanography Data Management Office}
\newacronym{WGDC}{WGDC}{Working Group on Data Citation}
\newacronym{COS}{COS}{Center for Open Science}
\newacronym{DCC}{DCC}{Digital Curation Centre}
\newacronym{CODATA}{CODATA}{Committee on Data of the International Council for Science}
\newacronym{ACCESS}{ACCESS}{Advancing Collaborative Connections for Earth System Science}
\newacronym{LPDAAC}{LPDAAC}{Land Processes Distributed Active Archive Center}
\newacronym{USGS}{USGS}{United States Geological Survey}
\newacronym{DFG}{DFG}{Deutsche Forschungsgemeinschaft}
\newacronym{ORNL}{ORNL}{Oak Ridge National Laboratory}
\newacronym{ARM}{ARM}{Atmospheric Radiation Measurement}
\newacronym{NITRC}{NITRC}{Neuroimaging Informatics Tools and Resources Clearinghouse}
\newacronym{CCCA}{CCCA}{Data Centre at the Climate Change Centre Austria}


% Concepts
\newacronym{GT}{GT}{gross tonnage}
\newacronym{UTC}{UTC}{Coordinated Universal Time}
\newacronym{FAIR}{FAIR}{Findable, Accessible, Interoperable, and Reusable}
\newacronym{PID}{PID}{Persistent Identifier}
\newacronym{UNF}{UNF}{Universal Numerical Fingerprint}
\newacronym{URL}{URL}{Uniform Resource Locator}
\newacronym{LSID}{LSID}{Life-Science Identifier}
\newacronym{URN}{URN}{Uniform Resource Names}
\newacronym{HNR}{HNR}{Handle.Net Registry}
\newacronym{CUD}{CUD}{Create, Update, and Delete}
\newacronym{ARK}{ARK}{Archival Resource Keys}
\newacronym{PURL}{PURL}{Persistent Uniform Resource Locators}
\newacronym{JDDCP}{JDDCP}{Joint Declaration of Data Citation Principles}
\newacronym{DEM}{DEM}{Digital Elevation Model}
\newacronym{DAAC}{DAAC}{Distributed Active Archive Center}
\makeglossary


\addbibresource{../../library/snow.bib} %Imports 
\addbibresource{../../library/scidb.bib} %Imports 

\title{PhD dissertation pre-proposal}
\author{Niklas Griessbaum}
\date{2018-05-19}


\begin{document}
\maketitle
%\printglossaries


\section{Motivation}

Environmental informatics is the application of information science to environmental sciences.
As such, it addresses the information infrastructure that environmental scientists leverage
to obtain knowledge form environmental data.

The appearance of cloud computing, characterized by scalability on one side, 
and by abstraction and stereotyping of interactions through \glspl{API} on the other side, 
provides for opportunities in environmental informatics. 
At the same time, it opens up questions of what the workflow
of environmental scientists should look like to fully exploit these opportunities.

My thesis is that the inherent heterogeneity of the flow from data to knowledge
in the environmental sciences causes bottlenecks that can be unblocked through 
infrastructures that are leveraging cloud computing.

In my dissertation, I want to address the dusk of file centric-data access.

Files package (i.e. chunk or aggregate) data into logical units and provide an intelligible identity to their content: Their filename.
In a file-centric data access scheme, users extract, transform, and load 
files prior to extracting knowledge.

Traditional file-centric data access thus results in three distinct bottlenecks:
\begin{enumerate}
 \item Files have to be moved prior to knowledge extraction.
 \item The pre-defined package size of files results in a transfer overhead.
 \item Alignment of heterogeneous data has to be handled manually.
\end{enumerate}

Files are moved (really: copied) before analysis for two reasons: 
1: The origin of the data (i.e. the repositories) does not offer any, or not sufficient capabilities to execute calculations on the data. Data, therefore,
has to be moved to the point of computation. 
2: The analysis requires the use of data from various sources. Data, therefore, is moved to integrate different datasets. \\
Data movement is undesired since it results in unstructured duplication of data.
The more distressing reason though is that compute (DRAM) bandwidths are exceeding effective network bandwidths by an order of magnitude. The trends of the last 20 years hereby exhibit increasing disparity.


The pre-defined package size of files will in a lot of cases not equal the area of
interest for the analysis. A file might for example package every band of a
sensor while only one band is of interest for the analysis. As long as files are the smallest subset unit for extraction, a transfer \textbf{overhead} will arise if  
the area of interest does not fully cover the pre-defined package size of the file.

Data have to be \textbf{aligned} manually when no common method to address tempo-spatial coincidence exists throughout the datasets used in an analysis. This is typically 
the case when data from different sources are integrated. Alignment is then manually achieved through projection, aggregation, and gridding of all datasets into a shared schema.

Through the use of common indexes, the alignment of heterogeneous data can be simplified.
Further, technologies like \gls{OPeNDAP} reduce the data transfer overhead by allowing
sub-setting of files prior to transfer.\footnote{The combination of both of these approaches is a subject of the proposal ``SpatioTemporal Adaptive-Resolution Encoding to Unify Diverse Earth Science Data for Integrative Analysis'' submitted to ``Advancing Collaborative Connection for Earth System Science'' (NNH17ZDA001N-ACCESS).}
However, the very necessity to move data prior to knowledge extraction remains intact
as long as computations are not executed at the place of storage.

\newpage

\section{Work Plan}
On the path to a work-flow in which knowledge can be extracted from data without the necessity of prior data movement, following requirements have to be fulfilled:

\paragraph{Co-location of data:}
Data has to be moved before analysis to integrate data from different sources. 
To avoid this data movement, all required data have to be readily co-located at the place of computation. In practical terms, this means storage of data in databases
rather than in files.

\paragraph{Alignment of data:}
The integration of heterogeneous data requires data to be aligned prior to analysis. I.e. a common concept to address spatiotemporal coincidence needs to be established throughout all the data.

\paragraph{Provision of data identity:}
By abandoning files as the package of data, a natural identity of data is lost. Identity, however, is needed to refer to, and reason about data.


\subsection{Data Citation}
The addressable identity of data is lost as soon as files as an intelligible package of data are abandoned.
Data identity, however, is needed to refer to and reference data, and moreover to provide provenance. Provenance, in turn, allows the capability to reproduce work,
which ensures trust in the data. 
Trust finally is required for researchers to build on top of each other's work.

Data citations can provide identity to data outside of file-centric environments. They can serve as an interface of trust in a heterogeneous system. 

%In order to create seamless workflows, data citations have to be machine actionable, both in terms of creation and in terms of resolving.

Data citation differ from citations of printed material 
in that the cited content (i.e. the data) may evolve over time and in that meta-information such as authorship or the provenance may vary within a continuous dataset. Further, since infinite ways to subset data are possible, data citations cannot be statically generated. They have to be machine-actionable, both in terms of dynamic creation (as a function of time and subsetting parameters) and in terms of resolving citations back to the cited material.

I want to address the challenges of data citations by elaborating on the following questions in a data citation review:

\begin{itemize}
    \item   How is data cited as of today? 
    \item   What is the status of machine-actionable data citation? 
    \item   How are data citations resolved?
    \item   What are the efforts of data citation standardization? 
    \item   What information needs to be available to create a data citation?
    \item   What information does a data citation need to contain?
    \item   What rules are needed to create data citations?    
    \item   How do the approaches on data citations employed by e.g. schema.org, Datalite,  EasyID, RDA, DataCite, or DataOne equal and differ?    
    \item   What is the distinguishing difference between a data citation and a \gls{DOI}.
\end{itemize}

On the practical side, I am addressing data citation through
the implementation of the OPeNDAP extension \textit{bibdap}\footnote{Former name: OCCUR}.
With the development of \textit{bibdap}, I want to demonstrate how a data provision service can be augmented with the capability to automatically generate and resolve machine actionable data citations subject to the \gls{DAS} (static metadata), subsetting parameters and the time of creation. The development of \textit{bibdap} is supported by the ESIP federation as a 2018 \gls{ESIP} lab project. The concept of \textit{bibdap} has been presented at the 2017 Bren PhD Symposium.
  
\newpage

\subsection{Data co-location and co-alignment}
In order to relax the necessity to move data prior to knowledge generation, 
all datasets required for analysis have to be readily stored at the place of computation.
The co-located datasets further need to be co-aligned
to allow efficient interoperability. 
Co-alignment means that spatiotemporal coincidence can be addressed
in the same way for all datasets. Co-alignment can be achieved through a shared
indexing schema. 
In the most simple form, this could be a regular grid. 
A side-effect of co-alignment is that it can be exploited to improve physically data placement: If spatiotemporally coincidental data is stored in physical proximity, 
it can be processed (embarrassingly) parallel in a shared-nothing architecture.

Building on top of the existing technologies of SciDB and the \gls{STARE}\footnote{STARE is a global indexing schema that is built upon \gls{HTM}.}, 
I started to develop a system that co-locates and co-aligns data at the place of computation. I have presented the findings of an early stage of a system that indexes and stores MODIS-swath data in a SciDB cluster during an oral presentation at the AGU Fall meeting 2017.

I want to continue the development of this system to address the following questions.
\begin{itemize}
 \item How does a system that leverages STARE within SciDB compare to google earth engine and the SpatioTemporal Adaptive Resolution Encoding?
 \item How should SciDB schemas be designed to store STARE-indexed data for various sensors (e.g. MODIS, VIRRS, Landsat, point measurements, ASO)?
 \item How can data from a variety of sensors be efficiently STARE-indexed and loaded into SciDB?
 \item How can SciDB chunking sizes be optimized for data of unknown sparseness?
 \item What are the relevant criteria to decide on the indexing method for (swath) pixels? (Pixels can e.g. be indexed through tessellation or through indexing of the centroids).
 \begin{itemize}    
    \item How can we quantify the loss of accuracy for the different methods?
    \item How can the storage and compute overhead for pixels indexed through tessellation be quantified?
 \end{itemize} 
 \item How can (STARE-)SciDB be operated in a queue manager driven \gls{HPC} environment.
 \item How can a file-view (e.g. \gls{HDF}) onto data stored in SciDB be realized for purposes of legacy compatibility.
 \item How can the interactions with STARE-indexed data stored in SciDB be abstracted from high level programming languages. \footnote{Reference to the 
 anticipated capabilities of STARS\footnote{\url{https://github.com/r-spatial/stars}} and openEO}
\end{itemize}

\subsection{Use case}
A compute system that co-locates and co-aligns datasets has its greatest utility for
analysis of large amounts of heterogeneous data; i.e. if data of various shapes are to be integrated. An interesting and apt example is the calculation and comparison of sub-pixel snow cover and snow temperature from various observations.
Subpixel snow cover can be calculated from satellites that measure in two or more thermal bands. These include MODIS, VIIRS, Landsat, GOES.
At the same time. a variety of other sources of information on snow cover exist, such as snow-tracks, snow-pillows or the \gls{ASO}.
Each one of these data has widely different spatial and temporal resolutions.

The research question that I therefore want to answer in this context is:
Can the effort to extract knowledge from these heterogeneous data sources be reduced
if data is co-located in a SciDB cluster and co-aligned through STARE prior to analysis.




%\section{Literature review}:
%
%\cite{Dozier1981} presents a method to measure surface radiant temperatures at sub-pixel
%resolution. It exemplifies a method for TIROS-N satellites.
%The fundamental principle is that a sub-pixel area with higher temperature will contribute
%proportinally more to the signal in the shorter wavelengths.
%The method assumes that only two temperature fields exist within the pixel.
%
%\Cite{Painter2009} present a method to obtain the fraction of each MODIS pixel that is covered by snow. Simultaneously, grain size of the fractional snow cover is estimated. Combined with an estimation for snow impurities, the albedo of the snow can be retrieved.
%The underlying method is spectral unmixing. Spectral unmixing assumes that the radiance received at the sensor is a linear combination of reflected radiances from the different surfaces (endmembers). In order to solve for the fractional surface types, the error of linear combination is minimized through a least-square fit.
%Endmembers of pure surface covers for varying grain sizes and zenith angles are pre-calculated. Additionally, endmembers for rock, soil, vegetation, and ice are calculated.
%
%\cite{Lundquist2018} presents a method to separate snow from forest temperatures and to determine fractional snow cover from MODIS data.
%A night time approach and a day time approach is provided.
%%MOD09GA: visible for forest cover (sinosoidal) *MODIS 1B for infrared (swath)


%\section{Reading List}
\printbibliography 

\end{document}
